\documentclass[11pt]{article}
\usepackage[colorlinks = true]{hyperref}
\usepackage{geometry}                % See geometry.pdf to learn the layout options. There are lots.
\geometry{letterpaper}                   % ... or a4paper or a5paper or ... 
%\geometry{landscape}                % Activate for for rotated page geometry
%\usepackage[parfill]{parskip}    % Activate to begin paragraphs with an empty line rather than an indent
\usepackage{graphicx}
\usepackage{amssymb}
\usepackage{epstopdf}
\DeclareGraphicsRule{.tif}{png}{.png}{`convert #1 `dirname #1`/`basename #1 .tif`.png}

\title{ADS2 Assessed Exercise: Modeling a Cache -- Part II}
\author{Wim Vanderbauwhede}
\date{\vspace{-5ex}}                                           % Activate to display a given date or no date

\begin{document}
\maketitle

\section{Aim}\label{aim}

The aim of this coursework is to create a Java model for a simple cache as used in modern computer systems.

\section{What to submit}\label{what-to-submit}

For the second part of the assignment, you have to write the code for the \texttt{FullyAssocLiFoCache} class based on the provided skeleton and the required Java datastructures.
\begin{itemize}
\item You \emph{must} start from the provided code and you should \emph{only add your own code} in \texttt{FullyAssocLiFoCache.java}, you \emph{must} not modify or removed any part of the provided code. 
\item You \emph{must} use the \texttt{Status} class for status reporting as it will be used by the testbench to verify the correctness of your code.
\item If you use \texttt{println } statements in your code, they \emph{must} be guarded by an \texttt{if(VERBOSE)} guard, for example:
\begin{verbatim}
   if (VERBOSE) System.out.println("Flushing cache");
\end{verbatim}

\item You \emph{must} submit this code in a \emph{gzipped tar archive} (other formats will \emph{not} be accepted) through
the Moodle submission system, and the filename \emph{must} be
\emph{\textless{}your matric + 1st char of your name in lowercase\textgreater{}.tgz}, so for example if your matric number is \emph{1107023m}
then your archive \emph{must} be named \emph{1107023m.tgz}. This archive
\emph{must} contain a single folder named \emph{\textless{}your matric + 1st char of your name in lowercase\textgreater{}}. This folder
\emph{must} contain a subfolder \texttt{src/ads2/cw1/} with following files:
\begin{itemize}
\item \texttt{Cache.java} as provided
\item \texttt{FullyAssocLiFoCache.java}  in which you must implement the functionality for the cache model
\item \texttt{Main.java} as provided
\item \texttt{Memory.java} as provided
\item \texttt{Status.java} as provided
\item \texttt{TestBench.java} as provided
\end{itemize}
\end{itemize}

\section{What is provided}\label{what-is-provided}

You can get the coursework source files and the coursework description at \url{https://github.com/wimvanderbauwhede/ADS2}. If any updates would be required I will push them to this repository.


\section{How to test your code}\label{how-to-test-your-code}

You should unit test every method you implement. Once your code is fully implemented you can use the provided testbench to see if it works as expected.

\section{Marking}\label{marking}


\begin{itemize}
\item Please note that whenever the specification (this document) says \emph{must}, then you will lose marks if you don't. In particular, if you do not follow the required structure and naming conventions for your submission, you will be penalised.  

\item Your code will be marked using a test script. This script will test if your code builds correctly and runs correctly for a number of use cases (70\% of the marks). I will not modify the provided testbench code but I may modify the memory and cache parameters. I will also run an additional test. 
\item The script will also perform source code analysis to see if you have used the correct API calls, data structures etc (30\% of the marks).  In particular, I will check
\begin{itemize}
\item Correctness of the constructor of the FullyAssocLiFoCache class.
\item Correct use of the various attributes of the FullyAssocLiFoCache class in the implementation of its methods.
\item Correct implementation and use use of the various methods of the FullyAssocLiFoCache class.
\item Correct use of the Status class to report the status of the cache at every operation.
\end{itemize}
\end{itemize}

\end{document}  